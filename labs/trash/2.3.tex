\documentclass[a4paper,14pt]{extarticle}
\usepackage[T2A]{fontenc}
\usepackage[utf8]{inputenc}
\usepackage[english,russian]{babel}
\usepackage{amsmath,amsfonts,amsthm, mathtools}
\usepackage{amssymb}
\usepackage{icomma}
\usepackage{graphicx}
\usepackage{wrapfig}
\RequirePackage{longtable}
\usepackage{soulutf8} 
\usepackage{geometry}
\geometry{top=20mm}
\geometry{bottom=20mm}
\geometry{left=20mm}
\geometry{right=20mm}


\usepackage{cmap}					
\usepackage{mathtext} 				
			
		

\usepackage{multirow}
\usepackage{graphicx}
\usepackage{wrapfig}
\usepackage{tabularx}
\usepackage{float}
\usepackage{hyperref}
\hypersetup{colorlinks=true,urlcolor=blue}
\usepackage[rgb]{xcolor}
\usepackage{amsmath,amsfonts,amssymb,amsthm,mathtools} 
\usepackage{icomma} 
\usepackage{euscript}
\usepackage{mathrsfs}
\usepackage{enumerate}
\usepackage{caption}
\usepackage{enumerate}
\mathtoolsset{showonlyrefs=true}

\usepackage{caption}
\usepackage{subcaption}

\usepackage[europeanresistors, americaninductors]{circuitikz}
\DeclareMathOperator{\sgn}{\mathop{sgn}}
\newcommand*{\hm}[1]{#1\nobreak\discretionary{}
	{\hbox{$\mathsurround=0pt #1$}}{}}
\usepackage{lineno}
\begin{document}
	\section{Задача 2.3}
	В последовательности вещественных чисел $a_0,a_1,a_2 ... ,a_N$ первые два члена
	($a_0,a_1$) заданы, а последующие ($a_0,a_1,a_2 ... ,a_N$) определяются с помощью
	рекуррентного соотношения
	\newline
	\begin{center}
		$a_{n+1}+a_{n-1} = \arcsin 2a_{n}$
	\end{center}
	Последовательность обрывается в тот момент, когда дальнейшее применение этого
	соотношения становится невозможным ($|a_N | > 1/2$, так что арксинус
	неопределен). Таким образом, длина последовательности N($a_0,a_1$) полностью
	определяется ее начальными условиями (т.е., первыми двумя членами).
	\newline
	1. Покажите, что при некоторых особых начальных условиях ($a_{0}^{*},a_{1}^{*}$)
	последовательность бесконечна: $N(a_{0}^{*},a_{1}^{*}) = \infty$. Как выглядит множество
	особых точек {($a_{0}^{*},a_{1}^{*}$)} на плоскости начальных условий ($a_0,a_1$)? Что
	это: одна точка, множество изолированных точек, линии, целые области?
	\newline
%	Ответ: одна точка (0,0)
	\newline
	2. Как выглядят достаточно далекие члены в бесконечной последовательности?
	\newline
%	Ответ: арифметическая прогрессия или тождественный ноль
	\newline
	3. По какому закону расходится величина N($a_0,a_1$) при приближении
	точки ($a_0,a_1$) к какой-либо особой точке ($a_{0}^{*},a_{1}^{*}$)?
	\newline
%	Ответ: $N \propto \mid \frac{\frac{1}{2}-a_0}{a_1-a_0} \mid$
	\newline
	4. Опишите далекие члены в очень длинной, но конечной последовательности
	($N \gg 1$)
	\newline
%	Ответ: $a_n \simeq a_0 + (a_1-a_0)n$
	\newline
	\begin{center}
		Решение
	\end{center}
	\linenumbers
	$\frac{d}{dx}(\arcsin x - x) = \frac{1}{\sqrt{1-x^2}} - 1 \geq 0$ при $0 \leq x <1$ и $\arcsin 0 - 0 = 0$, значит $ \arcsin x \geq  x $ при $0 \leq x <1$
	\newline
	Также $ \arcsin x \leq  x $ при $-1 < x \leq 0$
	\newline
	Тогда $\mid \arcsin x \mid \geq \mid x \mid$
	\newline
	Пусть $\exists N: 
	\begin{cases}
		a_N < 0
		\\
		a_{N-1} \geq 0
	\end{cases}$
	\newline
	Тогда $a_{N+1} = \arcsin(2a_N) - a_{N-1} \leq 2a_N - a_{N-1} < a_N < 0 $
	\newline
	$a_{N+2} = \arcsin(2a_{N+1}) - a_{N} \leq 2a_{N+1} - a_{N} = a_{N+1} + (a_{N+1}- a_{N}) < a_{N+1} < a_N < 0 $
	\newline
	Мы только что показали, что $a_{N+1} < a_N < 0 \Rightarrow a_{N+2} < a_{N+1} < a_N < 0$
	\newline
	Тогда $\forall n \geq N \; a_{n+1} < a_n$, то есть $\{a_n\} \downarrow$
	\newline
	Значит либо $\{a_n\}$ обрывается, либо имеет конечный предел: $-\frac{1}{2} \leq \lim\limits_{n \rightarrow \infty} a_n < 0$
	\newline
	Предположим, что предел существует и конечен и равен $x$. 
	\newline
	Тогда $x+x = \arcsin 2x \Rightarrow x = 0$. Противоречие. Значит $\{a_n\}$ обрывается.
	\newline
	Пусть $\exists N: 
	\begin{cases}
		a_N < 0
		\\
		a_{N-1} \geq a_N
	\end{cases}$
	\newline
	Тогда $a_{N+1} = \arcsin(2a_N) - a_{N-1} < 2a_N - a_{N-1} \leq a_N < 0 $
	\newline
	$a_{N+2} = \arcsin(2a_{N+1}) - a_{N} < 2a_{N+1} - a_{N} = a_{N+1} + (a_{N+1}- a_{N}) \leq a_{N+1} < a_N < 0 $
	\newline
	Мы только что показали, что $a_{N+1} < a_N < 0 \Rightarrow a_{N+2} < a_{N+1} < a_N < 0$
	\newline
	Тогда $\forall n \geq N \; a_{n+1} < a_n$, то есть $\{a_n\} \downarrow$
	\newline
	Значит либо $\{a_n\}$ обрывается, либо имеет конечный предел: $-\frac{1}{2} \leq \lim\limits_{n \rightarrow \infty} a_n < 0$
	\newline
	Предположим, что предел существует и конечен и равен $x$. 
	\newline
	Тогда $x+x = \arcsin 2x \Rightarrow x = 0$. Противоречие. Значит $\{a_n\}$ обрывается.
	\newline
	Иначе $\forall N \; 
	\left[
	\begin{gathered} 
		a_N \geq 0 \hfill
		\\
		a_{N-1} < a_N \hfill
		\\
	\end{gathered}
	\right.$
	\newline
	Пусть $\exists N: a_N < 0$. Тогда $0 >  a_N > a_{N-1}$.
	\newline
	$a_{N-1} < 0 \Rightarrow a_{n-1} > a_{N-2}$.
	\newline
	Значит  $\forall n \leq N \; 0 > a_n > a_{n-1} > \dots > a_0$, то есть $\{a_n\} \uparrow$ до $n = N$
	\newline
	$A = \{ N | a_N < 0\}$. Предположим, что $A$ конечно. Возьмем $M = \sup A$.
	\newline
	Тогда $\forall m \leq M \; \{a_m\} \uparrow$ и $a_m < 0$ 
	\newline
	$\forall n > M \;  a_n \geq 0$
	\newline
	$a_{M+1} > a_M$
	\newline
	$a_{M+2} = \arcsin(2a_{M+1}) - a_{M} \geq 2a_{M+1} - a_{M} = a_{M+1} + (a_{M+1}- a_{M}) > a_{M+1}$
	\newline
	$a_{M+3} = \arcsin(2a_{M+2}) - a_{M+1} \geq 2a_{M+2} - a_{M+1} = a_{M+2} + (a_{M+2}- a_{M+1}) > a_{M+2}$
	\newline
	Тогда $\{a_m\} \uparrow$ при $m \geq M$
	\newline
	Значит либо $\{a_m\}$ обрывается, либо имеет конечный предел: $0 < \lim\limits_{n \rightarrow \infty} a_n \leq \frac{1}{2}$
	\newline
	Но если этот предел существует, то он равен $0$. Противоречие. Значит $\{a_m\}$ обрывается.
	\newline
	Тогда осталось 2 варианта:
	\newline
	$1)\; \forall n \; a_n \geq 0$
	\newline
	Пусть $\exists N: a_N > a_{N-1}$
	\newline
	Тогда $a_{N+1} = \arcsin(2a_N) - a_{N-1} \geq 2a_N - a_{N-1}  = a_{N} + (a_{N}- a_{N-1}) > a_N $
	\newline
	Аналогичными рассуждениями приходим к тому, что $\{a_n\} \uparrow$ и последовательность обрывается.
	\newline
	Иначе $ \forall n \; a_n \leq a_{n-1}$
	\newline
	Тогда $\{a_n\} \downarrow$ и ограничена снизу, значит если такая ситуация возможна, то по т. Вейерштрасса $\lim\limits_{n \rightarrow \infty} a_n = 0$
	\newline
	$2)\; \forall N 	
	\begin{cases}
		a_N < 0
		\\
		a_{N-1} < a_N
	\end{cases}$
	\newline
	Аналогичными рассуждениями приходим к тому, что $\{a_n\} \uparrow$ и ограничена сверху, значит если такая ситуация возможна, то по т. Вейерштрасса $\lim\limits_{n \rightarrow \infty} a_n = 0$
	\newline
	Теперь поймем, возможны ли вообще последние 2 случая.
	\newline
	При больших $n \; a_n \ll 1 \Rightarrow a_{n+1}+a_{n-1} = \arcsin 2a_{n} \sim 2a_n$
	\newline
	 $a_{n+1}-a_{n} = a_{n} - a_{n-1} \Rightarrow \{a_n\} - \text{арифметическая прогрессия}$
	\newline
	Значит если $\; \forall n \; a_n \neq a_{n+1}$, то $\{a_n\}$ расходится(обрывается), так как $a_n \simeq a_0 + (a_1-a_0)n$ и $\exists N: \mid a_N \mid > \frac{1}{2}$.
	\newline
	Тогда единственный возможный вариант $a_n \equiv 0$
	\newline
	2. Достаточно далекие члены в бесконечной последовательности выглядят как арифметическая прогрессия или тождественный ноль
	\newline
	3. При приближении точки ($a_0,a_1$) к нашей особой точке ($0,0$) несложно оценить, по какому закону расходится величина N($a_0,a_1$):  
	\newline
	$a_N \simeq a_0 + (a_1-a_0)N = \frac{1}{2} \Rightarrow N \propto \mid \frac{\frac{1}{2}-a_0}{a_1-a_0} \mid$
	\newline
	4. Далекие члены в очень длинной, но конечной последовательности($N \gg 1$):
	\newline
	$a_n \simeq a_0 + (a_1-a_0)n$
	\newline
	(Тут используем формулу для малых $a_n$, так как иначе последовательность оборвется довольно быстро)
	 

\end{document}