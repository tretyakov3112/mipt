\documentclass[a4paper,14pt]{extarticle}
\usepackage[T2A]{fontenc}
\usepackage[utf8]{inputenc}
\usepackage[english,russian]{babel}
\usepackage{amsmath,amsfonts,amsthm, mathtools}
\usepackage{amssymb}
\usepackage{icomma}
\usepackage{graphicx}
\usepackage{wrapfig}
\RequirePackage{longtable}
\usepackage{soulutf8} 
\usepackage{geometry}
\geometry{top=20mm}
\geometry{bottom=20mm}
\geometry{left=20mm}
\geometry{right=20mm}
% новая команда \RNumb для вывода римских цифр
\newcommand{\RNumb}[1]{\uppercase\expandafter{\romannumeral #1\relax}}

\usepackage{cmap}					
\usepackage{mathtext} 				
			
		

\usepackage{multirow}
\usepackage{graphicx}
\usepackage{wrapfig}
\usepackage{tabularx}
\usepackage{float}
\usepackage{hyperref}
\hypersetup{colorlinks=true,urlcolor=blue}
\usepackage[rgb]{xcolor}
\usepackage{amsmath,amsfonts,amssymb,amsthm,mathtools} 
\usepackage{icomma} 
\usepackage{euscript}
\usepackage{mathrsfs}
\usepackage{enumerate}
\usepackage{caption}
\usepackage{enumerate}
\mathtoolsset{showonlyrefs=true}

\usepackage{caption}
\usepackage{subcaption}

\usepackage[europeanresistors, americaninductors]{circuitikz}
\DeclareMathOperator{\sgn}{\mathop{sgn}}
\newcommand*{\hm}[1]{#1\nobreak\discretionary{}
	{\hbox{$\mathsurround=0pt #1$}}{}}
\usepackage{lineno}
\begin{document}
	\section{Задача 4.1}
	Если некоторое выпуклое трехмерное тело спроектировать на плоскость,
	характеризуемую нормалью n, то площадь получившейся проекции будет
	равна S(n). Выразите среднее по направлениям нормали $<S(\vec{n})>_{\vec{n}}$ через интегральные
	характеристики тела (например, такие, как объем, площадь поверхности, ее
	средняя кривизна, наибольшее или наименьшее сечение, и т.п.)
	\newline
	Ответ: $<S(\vec{n})>_{\vec{n}} = \frac{S}{4}$, где S - площадь поверхности тела.
	\begin{center}
		Решение
	\end{center}
	\linenumbers
	$S(\vec{n}) = \int_{D}^{} \vec{n} d\vec{S} = \int_{D^{*}}^{} \vec{n^{*}} d\vec{S} = S(\vec{n^{*}})$, где $\vec{n^{*}} = -\vec{n} $, 
	\newline
	а $D \; \text{и} \; D^{*} - \text{противоположные части поверхности тела}$
	\newline
	$2S(\vec{n}) = S(\vec{n})+S(-\vec{n}) = \oint \vec{n} d\vec{S} $ 
	\newline
	Рассмотрим площадку S с единичным вектором нормали $\vec{N}$
	\newline
	Тогда $S(\vec{n}) = \mid \vec{n} \cdot \vec{N} \mid S$
	\newline
	$<S(\vec{n})>_{\vec{n}} = \frac{1}{4\pi} \int S(\vec{n}) \: \mid d\Omega \mid  =\frac{1}{4\pi} \iint S(\vec{n}) \mid \cos \theta \: d\phi \: d\theta \mid , \phi \in (0, 2\pi) , \theta \in (-\frac{\pi}{2}, \frac{\pi}{2}) $
	\newline
	$<S(\vec{n})>_{\vec{n}}  =\frac{1}{4\pi} \iint \mid \vec{n} \cdot \vec{N} \mid \cdot S \cdot \mid  \cos \theta \: d\phi \: d\theta \mid $
	\newline
	$\vec{n} = (\cos \theta \cos \phi , \cos \theta \sin \phi , \sin \theta)$
	\newline
	Так как мы усредняем по всем направлениям, 	$<S(\vec{n})>_{\vec{n}}$ не зависит от $\vec{N}$. Выберем $\vec{N} = (0, 0, 1)$
	\newline
	Тогда $<S(\vec{n})>_{\vec{n}}  =\frac{S}{4\pi} \iint \mid (\cos \theta \cos \phi , \cos \theta \sin \phi , \sin \theta) \cdot (0, 0, 1)   \cos \theta \: d\phi \: d\theta \mid $
	\newline
	$<S(\vec{n})>_{\vec{n}}  =\frac{S}{4\pi} \iint \mid  \sin \theta    \cos \theta \: d\phi \: d\theta \mid = \frac{S}{4\pi} \iint \mid  \sin \theta   \cos \theta \: d\phi \: d\theta \mid $
	\newline
	$<S(\vec{n})>_{\vec{n}} = \frac{S}{8\pi} \iint \mid  \sin 2\theta \: d\phi \: d\theta \mid  = \frac{S}{8\pi} \cdot 2\pi \cdot \frac{-2\cos 2\theta}{2} \bigg|_{0}^{\frac{pi}{2}} = \frac{S}{2}$
	\newline
	Разобьем поверхность тела на много маленьких площадок $d \vec{S}$
	\newline
	$<dS(\vec{n})>_{\vec{n}} = \frac{dS}{2}$
	\newline
	$\int <dS(\vec{n})>_{\vec{n}} = \int_S \frac{dS}{2} = \frac{S}{2}$
	\newline
	С другой стороны из 3 строчки следует, что $\int <dS(\vec{n})>_{\vec{n}} = 2 <S(\vec{n})>_{\vec{n}} $
	\newline
	Значит $<S(\vec{n})>_{\vec{n}} = \frac{1}{2} \cdot \frac{S}{2} = \frac{S}{4} $
	\newline
	Ответ: $<S(\vec{n})>_{\vec{n}} = \frac{S}{4}$
\end{document}