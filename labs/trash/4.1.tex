\documentclass[a4paper,14pt]{extarticle}
\usepackage[T2A]{fontenc}
\usepackage[utf8]{inputenc}
\usepackage[english,russian]{babel}
\usepackage{amsmath,amsfonts,amsthm, mathtools}
\usepackage{amssymb}
\usepackage{icomma}
\usepackage{graphicx}
\usepackage{wrapfig}
\RequirePackage{longtable}
\usepackage{soulutf8} 
\usepackage{geometry}
\geometry{top=20mm}
\geometry{bottom=20mm}
\geometry{left=20mm}
\geometry{right=20mm}


\usepackage{cmap}					
\usepackage{mathtext} 				
			
		

\usepackage{multirow}
\usepackage{graphicx}
\usepackage{wrapfig}
\usepackage{tabularx}
\usepackage{float}
\usepackage{hyperref}
\hypersetup{colorlinks=true,urlcolor=blue}
\usepackage[rgb]{xcolor}
\usepackage{amsmath,amsfonts,amssymb,amsthm,mathtools} 
\usepackage{icomma} 
\usepackage{euscript}
\usepackage{mathrsfs}
\usepackage{enumerate}
\usepackage{caption}
\usepackage{enumerate}
\mathtoolsset{showonlyrefs=true}

\usepackage{caption}
\usepackage{subcaption}

\usepackage[europeanresistors, americaninductors]{circuitikz}
\DeclareMathOperator{\sgn}{\mathop{sgn}}
\newcommand*{\hm}[1]{#1\nobreak\discretionary{}
	{\hbox{$\mathsurround=0pt #1$}}{}}
\usepackage{lineno}
\begin{document}
	\section{Задача 4.1}
	Если некоторое выпуклое трехмерное тело спроектировать на плоскость,
	характеризуемую нормалью n, то площадь получившейся проекции будет
	равна S(n). Выразите среднее по направлениям нормали $<S(\vec{n})>_{\vec{n}}$ через интегральные
	характеристики тела (например, такие, как объем, площадь поверхности, ее
	средняя кривизна, наибольшее или наименьшее сечение, и т.п.)
	\newline
	\begin{center}
		Решение(неоконченное)
	\end{center}
	\linenumbers
	$S(\vec{n}) = \int_{D}^{} \vec{n} d\vec{S} = \int_{D^{*}}^{} \vec{n^{*}} d\vec{S} = S(\vec{n^{*}})$, где $\vec{n^{*}} = -\vec{n} $, 
	\newline
	а $D \; \text{и} \; D^{*} - \text{противоположные части поверхности тела}$
	\newline
	$2S(\vec{n}) = S(\vec{n})+S(-\vec{n}) = \oint \vec{n} d\vec{S} = \int_{V}^{} div(\vec{n}) dV$ 
	\newline
	$\vec{n} = <\cos\theta, \; \sin\phi \cos\theta, \; -\cos\phi \sin\theta> $
	\newline
	$\vec{\nabla} = <\frac{\partial}{\partial r}, \; \frac{1}{r} \frac{\partial}{\partial \theta}, \; \frac{1}{r\sin\theta} \frac{\partial}{\partial \phi}>$
	\newline
	$\vec{\nabla} \cdot \vec{n} = \frac{1}{r} \frac{\partial}{\partial \theta} \sin\phi \cos\theta - \frac{1}{r\sin\theta} \frac{\partial}{\partial \phi} \cos\phi \sin\theta = \frac{\sin \phi}{r} (\cos \theta +1) $
	\newline
	$\int_{V}^{} div(\vec{n}) dV = \sin \phi (\cos \theta +1) \iiint r \sin \tilde \theta \: dr \: d \tilde{\phi} \: d \tilde{\theta} = \frac{\sin \phi (\cos \theta +1)}{2} \iint r^2(\tilde \theta , \tilde{\phi}) \sin \tilde \theta \: d \tilde{\phi} \: d \tilde{\theta}$
	\newline
	$I_{2n-1} = \iint r^2(\tilde \theta , \tilde{\phi}) \sin^{2n-1} \tilde \theta \: d \tilde{\phi} \: d \tilde{\theta} = \iint r^2(\tilde \theta , \tilde{\phi}) \sin^{2n+1} \tilde \theta \: d \tilde{\phi} \: d \tilde{\theta} +$
	\newline
	$+2 \iint \frac{1}{2}\tilde{r}^2(\tilde \theta , \tilde{\phi}) \sin^{2n-1} \tilde \theta \: d \tilde{\phi} \: d \tilde{\theta} = I_{2n+1} + 2 \int \sin^{2n-1} \tilde \theta \: S(\tilde \theta)  \: d \tilde{\theta}$, где $S(\tilde \theta)$ - сечение на уровне $\theta$
	\newline
	$I_{2n+1} = I_{2n-1} -2 \int \sin^{2n-1} \tilde \theta \: S(\tilde \theta)  \: d \tilde{\theta} = 
	I_{2n-3} -2 \int (\sin^{2n-1} \tilde \theta + \sin^{2n-3} \tilde \theta) \: S(\tilde \theta)  \: d \tilde{\theta} $
	\newline
	$I_{2n+1} = I_{1} -2 \int (\sin^{2n-1} \tilde \theta + \sin^{2n-3} \tilde \theta + \dots + \sin^{1} \tilde \theta) \: S(\tilde \theta)  \: d \tilde{\theta} = I_{1} -2 \int \sin \tilde \theta \frac{1 - \sin^{2n} \tilde \theta}{1 - \sin^{2} \tilde \theta} \: S(\tilde \theta)  \: d \tilde{\theta}$
	\newline
	$ \lim\limits_{n \rightarrow \infty} I_{2n+1} =  I_{1} -2 \int \sin \tilde \theta \lim\limits_{n \rightarrow \infty} \frac{1 - \sin^{2n} \tilde \theta}{1 - \sin^{2} \tilde \theta} \: S(\tilde \theta)  \: d \tilde{\theta}$
	\newline
	$0 = I_{1} -2 \int \frac{\sin \tilde \theta}{\cos^2 \tilde \theta} \: S(\tilde \theta)  \: d \tilde{\theta} \Rightarrow 
	I_{1} =  2 \int \frac{\sin \tilde \theta}{\cos^2 \tilde \theta} \: S(\tilde \theta)  \: d \tilde{\theta} =
	2 \int  S(\tilde \theta)  \: d \sec \tilde{\theta} $
	\newline
	$I_1 = 2  [S(\tilde \theta)  \sec \tilde{\theta} \bigg|_{-\frac{\pi}{2}}^{\frac{\pi}{2}} - \int \sec \tilde{\theta} \: dS(\tilde \theta)  ] = 2 [0 - \int \sec \tilde{\theta} \: dS(\tilde \theta)  ] = -2 \int \sec \tilde{\theta} \: dS(\tilde \theta)  $
	\newline
	$<S(\vec{n})>_{\vec{n}} = \frac{1}{4\pi} \int S(\vec{n}) \: d\Omega =\frac{1}{4\pi} \iint S(\vec{n}) \: d\phi \: d\theta = $
	\newline
	$= \frac{1}{4\pi} \cdot \frac{1}{2} \iint \frac{\mid \sin \phi \mid (\cos \theta +1)}{2} \: d\phi \: d\theta \cdot 2 \int  S(\tilde \theta)  \: d \sec \tilde{\theta} = \frac{2 \cdot 4 \cdot (\pi+2)}{4\pi \cdot 2 \cdot 2}  2 \int  S(\tilde \theta)  \: d \sec \tilde{\theta}  $
	\newline
		$<S(\vec{n})>_{\vec{n}} = \frac{\pi+2}{2\pi}  \int_{-\frac{\pi}{2}}^{\frac{\pi}{2}}  S(\tilde \theta)  \: d \sec \tilde{\theta} $
\end{document}