\documentclass[a4paper,14pt]{extarticle}
\usepackage[T2A]{fontenc}
\usepackage[utf8]{inputenc}
\usepackage[english,russian]{babel}
\usepackage{amsmath,amsfonts,amsthm, mathtools}
\usepackage{amssymb}
\usepackage{icomma}
\usepackage{graphicx}
\usepackage{wrapfig}
\RequirePackage{longtable}
\usepackage{soulutf8} 
\usepackage{geometry}
\geometry{top=20mm}
\geometry{bottom=20mm}
\geometry{left=20mm}
\geometry{right=20mm}


\usepackage{cmap}					
\usepackage{mathtext} 				
			
		

\usepackage{multirow}
\usepackage{graphicx}
\usepackage{wrapfig}
\usepackage{tabularx}
\usepackage{float}
\usepackage{hyperref}
\hypersetup{colorlinks=true,urlcolor=blue}
\usepackage[rgb]{xcolor}
\usepackage{amsmath,amsfonts,amssymb,amsthm,mathtools} 
\usepackage{icomma} 
\usepackage{euscript}
\usepackage{mathrsfs}
\usepackage{enumerate}
\usepackage{caption}
\usepackage{enumerate}
\mathtoolsset{showonlyrefs=true}

\usepackage{caption}
\usepackage{subcaption}

\usepackage[europeanresistors, americaninductors]{circuitikz}
\DeclareMathOperator{\sgn}{\mathop{sgn}}
\newcommand*{\hm}[1]{#1\nobreak\discretionary{}
	{\hbox{$\mathsurround=0pt #1$}}{}}

\begin{document}
	Ответ:
		$$y\sim C\,x^{1/3}e^{-\frac{3}{4}x^{2/3}} \sin \left (\frac{3\sqrt3}{4}x^\frac{2}{3}-\frac{\pi}{6} \right )$$
	Давайте проверим наш ответ с помощью Wolfram.
	$$f(x) = \frac{x y'''}{y} = x \frac{\dfrac{d^3}{dx^3} \bigg{[}C\,x^{1/3}e^{-\frac{3}{4}x^{2/3}} \sin \left (\frac{3\sqrt3}{4}x^\frac{2}{3}-\frac{\pi}{6} \right ) \bigg{]}}{C\,x^{1/3}e^{-\frac{3}{4}x^{2/3}} \sin \left (\frac{3\sqrt3}{4}x^\frac{2}{3}-\frac{\pi}{6} \right )}$$
	$$f(x) =  x \frac{\dfrac{d^3}{dx^3} \bigg{[}x^{1/3}e^{-\frac{3}{4}x^{2/3}} \sin \left (\frac{3\sqrt3}{4}x^\frac{2}{3}-\frac{\pi}{6} \right ) \bigg{]}}{x^{1/3}e^{-\frac{3}{4}x^{2/3}} \sin \left (\frac{3\sqrt3}{4}x^\frac{2}{3}-\frac{\pi}{6} \right )}$$
	$$f(1) = 1,77194$$
	$$f(10) = 0,935911$$
	$$f(10^2) = 0,996329$$
	$$f(10^3) = 1,00003$$
	$$f(10^4) = 0,999994$$
	%$$f(10^5) = 1,00003$$
	$$x y'''=y$$
	$$\mathfrak{F} \{ x y'''=y \}$$
	$$ i \frac{d}{d\nu} [ (i\nu)^3 \hat{y}] = \hat{y}$$
	$$ \frac{d}{d\nu} [ (\nu)^3 \hat{y}] = \hat{y}$$
	$$ 3\nu^2 \hat{y}+ \nu^3 \frac{d}{d\nu}\hat{y} = \hat{y}$$
	$$\frac{d\hat{y}}{\hat{y}} = \frac{1-3\nu^2}{\nu^3} d\nu$$
	$$\ln \hat{y} = -\frac{1}{2\nu^2} - 3 \ln\nu +C$$
	$$\hat{y} = \frac{c}{\nu^3} e^{-\frac{1}{2\nu^2}}$$
	$$y = c \int\limits_{-\infty }^{\infty }\frac{d\nu}{\nu^3\exp \left ( \frac{1}{2\nu^2}-ix\nu \right )}$$
	\begin{multline*}
		y=\int\limits_{-\infty }^{\infty }\frac{d\nu}{\nu^3\exp \left ( \frac{1}{2\nu^2}-ix\nu \right )}=\int\limits\limits_{-\infty}^{\infty }\frac{e^{-\frac{1}{2\nu^2}}}{\nu^3} e^{ix\nu}d\nu\overset{\nu=1/t}{=} \int\limits\limits_{-\infty}^\infty e^{-t^2/2+ix/t}\,t\,dt\overset{t=sx^{1/3}}{=}\\=x^{2/3}\int\limits\limits_{-\infty}^\infty e^{-x^{2/3}(s^2/2-i/s)}s\,ds
	\end{multline*}
	Используя метод крутого спуска, мы ищем точки, в которых
	$$\frac{d}{ds}\Big(\frac{s^2}{2}-\frac{i}{s}\Big)=0\Rightarrow s^3=-i$$
	У нас есть три такие точки: $\displaystyle s_1=e^{-\frac{\pi i}{6}}; s_2=e^{-\frac{5\pi i}{6}}; s_3=i$. Далее мы изменим контур интегрирования в комплексной плоскости следующим образом:
	$$-R\to-R-\frac{i}{2}\to R-\frac{i}{2}\to R;\, (R\to\infty)$$
	наша идея $-$ направить путь интегрирования через точки $s_{1,2}$. Внутри прямоугольного контура нет полюсов, поэтому, меняя путь, мы не меняем значение интеграла. Теперь вычисление становится простым. Например, для
	$$s_1=e^{-\frac{\pi i}{6}}\Rightarrow \frac{d^2}{ds^2}\Big(\frac{s^2}{2}-\frac{i}{s}\Big)\bigg|_{s=s_1}=3$$
	\begin{multline*}
		I_1(x)=x^{2/3}\int\limits_{-\infty}^\infty e^{-x^{2/3}(s_1^2/2-i/s_1)}s_1e^{-\frac{3}{2}x^{2/3}(s-s_1)^2}\,ds=\\= x^{2/3}e^{-x^{2/3}(3/4-i3\sqrt3/4)-\pi i/6}\int\limits_{-\infty}^\infty e^{-\frac{3}{2}x^{2/3}(s-s_1)^2}\,ds=\sqrt\frac{2\pi}{3}\,x^\frac{1}{3}e^{-\frac{3}{4}x^{2/3}}e^{i\big(\frac{3\sqrt3}{4}x^\frac{2}{3}-\frac{\pi}{6}\big)}
	\end{multline*}
	$$s_2=e^{-\frac{5\pi i}{6}}\Rightarrow I_2(x)=\sqrt\frac{2\pi}{3}\,x^{1/3}e^{-\frac{3}{4}x^{2/3}}e^{i\left (-\frac{3\sqrt3}{4}x^\frac{2}{3}-\frac{5\pi}{6} \right )}$$
	Тогда
	$$y\sim I(x)=I_1(x)+I_2(x)=2i\sqrt\frac{2\pi}{3}\,x^{1/3}e^{-\frac{3}{4}x^{2/3}}\sin \left (\frac{3\sqrt3}{4}x^\frac{2}{3}-\frac{\pi}{6} \right )$$
	Нашему ДУ удовлетворяет $y$ с точностью до константы
	\newline
	Сделаем вывод:
	$$y\sim C\,x^{1/3}e^{-\frac{3}{4}x^{2/3}} \sin \left (\frac{3\sqrt3}{4}x^\frac{2}{3}-\frac{\pi}{6} \right )$$
	Давайте проверим правильность асимптотики нашего интеграла с помощью Wolfram.
	$$A = \frac{x^{2/3}\int\limits\limits_{-\infty}^\infty e^{-x^{2/3}(s^2/2-i/s)}s\,ds}{2i\sqrt\frac{2\pi}{3}\,x^{1/3}e^{-\frac{3}{4}x^{2/3}} \sin \left (\frac{3\sqrt3}{4}x^\frac{2}{3}-\frac{\pi}{6} \right )} = \frac{x^{1/3}\int\limits\limits_{-\infty}^\infty e^{-x^{2/3}(s^2/2-i/s)}s\,ds}{2i\sqrt\frac{2\pi}{3}\,e^{-\frac{3}{4}x^{2/3}} \sin \left (\frac{3\sqrt3}{4}x^\frac{2}{3}-\frac{\pi}{6} \right )}   \overset{x^{1/3} = \lambda}{=}$$
	$$= \frac{\lambda\int\limits\limits_{-\infty}^\infty e^{-\lambda^2(s^2/2-i/s)}s\,ds}{2i\sqrt\frac{2\pi}{3}\,e^{-\frac{3}{4}\lambda^2} \sin \left (\frac{3\sqrt3}{4}\lambda^2-\frac{\pi}{6} \right )}$$
	$$A(\lambda = 5) = 1.021 $$
	$$A(\lambda = 6) = 0.998 $$
	$$A(\lambda = 7) = 1.018 $$
	$$A(\lambda = 8) = 1.004 $$
	$$A(\lambda = 9) = 1.003 $$
	$$A(\lambda = 10) = 1.005 $$
	Теперь давайте проверим, удовлетворяет ли наш ответ изначальному ДУ.
	$$f(x) = \frac{x y'''}{y} = x \frac{\dfrac{d^3}{dx^3} \bigg{[}C\,x^{1/3}e^{-\frac{3}{4}x^{2/3}} \sin \left (\frac{3\sqrt3}{4}x^\frac{2}{3}-\frac{\pi}{6} \right ) \bigg{]}}{C\,x^{1/3}e^{-\frac{3}{4}x^{2/3}} \sin \left (\frac{3\sqrt3}{4}x^\frac{2}{3}-\frac{\pi}{6} \right )}$$
	$$f(x) =  x \frac{\dfrac{d^3}{dx^3} \bigg{[}x^{1/3}e^{-\frac{3}{4}x^{2/3}} \sin \left (\frac{3\sqrt3}{4}x^\frac{2}{3}-\frac{\pi}{6} \right ) \bigg{]}}{x^{1/3}e^{-\frac{3}{4}x^{2/3}} \sin \left (\frac{3\sqrt3}{4}x^\frac{2}{3}-\frac{\pi}{6} \right )}$$
	$$f(1) = 1,77194$$
	$$f(10) = 0,935911$$
	$$f(10^2) = 0,996329$$
	$$f(10^3) = 1,00003$$
	$$f(10^4) = 0,999994$$
	%$$f(10^5) = 1,00003$$
\end{document}
