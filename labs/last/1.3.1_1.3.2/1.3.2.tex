\documentclass[a4paper,14pt]{extarticle}
\usepackage[a4paper,top=1.3cm,bottom=2cm,left=1.5cm,right=1.5cm,marginparwidth=0.75cm]{geometry}
\usepackage{setspace}
\usepackage{cmap}					
\usepackage{mathtext} 				
\usepackage[T2A]{fontenc}			
\usepackage[utf8]{inputenc}			
\usepackage[english,russian]{babel}
\usepackage{multirow}
\usepackage{graphicx}
\usepackage{wrapfig}
\usepackage{tabularx}
\usepackage{float}
\usepackage{longtable}
\usepackage{hyperref}
\hypersetup{colorlinks=true,urlcolor=blue}
\usepackage[rgb]{xcolor}
\usepackage{amsmath,amsfonts,amssymb,amsthm,mathtools} 
\usepackage{icomma} 
\mathtoolsset{showonlyrefs=true}
\usepackage{euscript}
\usepackage{mathrsfs}

\DeclareMathOperator{\sgn}{\mathop{sgn}}
\newcommand*{\hm}[1]{#1\nobreak\discretionary{}
	{\hbox{$\mathsurround=0pt #1$}}{}}

\usepackage[T2A]{fontenc}
\usepackage[utf8]{inputenc}
\usepackage[english,russian]{babel}
\usepackage{amsmath,amsfonts,amsthm, mathtools}
\usepackage{amssymb}
\usepackage{icomma}
\usepackage{graphicx}
\usepackage{wrapfig}
\RequirePackage{longtable}
\newcommand{\RomanNumeralCaps}[1]
{\MakeUppercase{\romannumeral #1}}

\usepackage{soulutf8} 
\usepackage{geometry}



\begin{document}
	\begin{center}
		\textit{Федеральное государственное автономное образовательное\\ учреждение высшего образования }
		
		\vspace{0.5ex}
		
		\textbf{«Московский физико-технический институт\\ (национальный исследовательский университет)»}
	\end{center}
	
	\vspace{10ex}
	

	\begin{center}
		\vspace{13ex}	
		\textbf{Лабораторная работа №1.3.2(2)}	
		\vspace{1ex}
	
		по курсу общей физики		
		на тему:		
		\textbf{\textit{<<Определение модуля кручения>>}}		
		\vspace{30ex}

	\begin{flushright}
		\noindent
		\textit{Работу выполнил:}\\  
		\textit{Санёчек Третьяков  \\(группа Б02-206)}
	\end{flushright}
	\vfill
	Маями Бич \\ \today

%\setcounter{page}{1}
\end{center}
\newpage
	\section{Аннотация}
	\textbf{Цель работы:} измерение углов закручивания в зависимости от приложенного момента сил, расчет модуля сдвига проволоки по измерениям периодов крутильных колебаний подвешанного на ней 	маятника(динамическим методом).\\
	\textbf{В работе используются:} проволока из исследуемого материала, грузы, секундомер, микрометр, рулетка, линейка.
		\section{Теоретическая справка}
	\begin{centering}
	Вращение описывается формулой:$$I_y\frac{d\omega}{dt} = M_y \Rightarrow I\frac{d^2\varphi}{dt^2} = -M$$
	,где I - момент инерции стержня, $\varphi$ - угол поворота стержня от положения равновесия, М - момент сил, действующий на стержень при закручивании; 
	\\При малых $\varphi$ момент сил описывается формулой: $$M = \frac{\pi R^4 G}{2l}\varphi = f\varphi \text{, где f модуль кручения, связанный с модулем сдвига G}$$
	Тогда получаем следующие уравнения для незатухающих колебаний: $$\frac{d^2\varphi}{dt^2} + \frac{f}{I} = 0 \Rightarrow \varphi = \varphi_0 sin(\sqrt{\frac{f}{I}}t + \theta) \Rightarrow T = 2\pi \sqrt{\frac{f}{I}}$$
	Таким образом модуль сдвига G выражается через $f$ следующей формулой: $$G = \frac{2lf}{\pi R^4}, \text{где  } R = \frac{d}{2}$$
	$$f = \frac{4\pi^2 I}{T^2} \Leftrightarrow T^2 = \frac{4\pi^2 I}{f}$$
	$$I = m_1r_1^2 + m_2r_2^2 +I_0 = 2mr^2 + I_0$$
	$$T^2 = \frac{8\pi^2m}{f}r^2 + \frac{4\pi^2 I_0}{f}$$
	Можно заметить, что зависимость $T^2(r^2)$ имеет вид $y = kx + b$, причем $$k = \frac{8\pi^2m}{f}\Rightarrow f = \frac{8\pi^2m}{k}, \;\;\;\; G = \frac{16\pi l m}{ k R^4} $$
	$$\sigma_G = G\sqrt{(\frac{\sigma_f}{f})^2 + (\frac{\sigma_l}{l})^2 + 16(\frac{\sigma_d}{d})^2}$$
	\end{centering}
\newpage
	\section{Методика измерений}
	Экспериментальная установка состоит из исследуемой проволоки "П" и прикрепленного к ее нижнему концу стержня "С" с двумя симметрично расположенными грузами "Г". Верхний конец проволоки может спокойно проворачиваться вокруг вертикальной оси.
	Установка для определения модуля сдвига проволочки представлена на рисунке ниже:
	\begin{figure}[H]
		\begin{center}
			\includegraphics[scale = 0.35]{"ustan_4.jpg"}
			\caption{схема установки}
		\end{center}
	\end{figure}
	\section{Результаты эксперимента}
	Массы грузиков: $m = (337\pm 1)$г.
	\\Длина проволочки: $l = (171,6 \pm 0,1)$см
	\begin{table}[H]
		\centering
		\begin{tabular}{|l|l|l|l|l|l|l|l|l|l|l|l|}
			\hline
			№ & 1 & 2 & 3 & 4 & 5 & 6 & 7 & 8 & 9 & 10 & Среднее \\ \hline
			d, мм & 1,56 & 1,56 & 1,55 & 1,55 & 1,55 & 1,55 & 1,55 & 1,55 & 1,55 & 1,55 & 1,552 \\ \hline
		\end{tabular}
	\caption{эксперимнтальные данные d}
	\end{table}
	Диаметр проволочки: $d_\text{ср} = (1,552 \pm 0,011)$мм.
	\begin{table}[!ht]
		\centering
		\begin{tabular}{|l|l|l|l|l|l|}
			\hline
			r, м & 0,205 & 0,210 & 0,215 & 0,220 & 0,225 \\ \hline
			T, с & 5,178 & 5,294 & 5,41 & 5,524 & 5,674 \\ \hline
			$r^2, \text{м}^2$ & 0,042 & 0,044 & 0,046 & 0,048 & 0,051 \\ \hline
			$T^2, \text{с}^2$ & 26,8 & 28,02 & 29,27 & 30,51 & 32,19 \\ \hline
		\end{tabular}
	\caption{экспериментальные данные $T$ и посчитанные $T^2$}
	\end{table}
	\section{Обработка результатов эксперимента}
	По полученным из эксперимента данным построим график $T^2(r^2)$:
	\begin{figure}[H]
		\begin{center}
			\includegraphics[scale = 0.6]{"похуй.jpg"}
			\caption{схема установки}
		\end{center}
	\end{figure}
	1.3.2 здесь вроде все нормально получилось(погрешность из воздуха взял)
	Из графика:  $k = 628,22 \Rightarrow G = 7,6*10^{10} \;\text{Па} \;\;\;\;\sigma_G = 0,6 * 10^{10}$
	\section{Выводы}
	Проволока скорее всего сделана из стали. Табличное значение модуля сдвига стали 83 ГПа. Экспериментально полученный модуль сдвига - 76 $\pm$ 6 ГПа. Отличие от табличного значения составило $\varepsilon(G) = 8\%$
	Это можно объяснить тем, что проволока сделана из сплава или с добавлением примесей.
\end{document}